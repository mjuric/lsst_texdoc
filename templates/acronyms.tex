The following table has been generated from the on-line Gaia acronym list:
\newline\newline%decrement table counter so table sin doc start at 1.
\addtocounter{table}{-1}
\begin{longtable}{|l|p{0.8\textwidth}|}\hline 
\textbf{Acronym} & \textbf{Description}  \\\hline
AGIS&Astrometric Global Iterative Solution \\\hline
AO&Announcement of Opportunity \\\hline
AS&Adjacent Sample \\\hline
ATP&Automatic Test Procedure \\\hline
AUT&AUTomated \\\hline
CCB&Configuration Control Board \\\hline
CDR&Critical Design Review \\\hline
CIL&Critical Items List \\\hline
CM&Calibration Model \\\hline
CN&Change Notice \\\hline
CNES&Centre National d'Etudes Spatiales (France) \\\hline
COTS&Commercial-Off-The-Shelf \\\hline
CPU&Central Processing Unit \\\hline
CRB&Change Review Board \\\hline
CRR&Command Request Response \\\hline
CSV&Comma-Separated Value (database output format, e.g., for MS Excel) \\\hline
CU&Coordination Unit (in DPAC) \\\hline
DB&DataBase \\\hline
DDP&Delivered Duty Paid \\\hline
DOC&Department of Commerce (USA) \\\hline
DPAC&Data Processing and Analysis Consortium \\\hline
DPC&Data Processing Centre \\\hline
DPCE&Data Processing Centre ESAC \\\hline
DPCG&Data Processing Centre (ObsGE/ISDC) Geneva \\\hline
DU&Development Unit (in DPAC) \\\hline
ECSS&European Cooperation for Space Standardisation \\\hline
ESA&European Space Agency \\\hline
ESAC&European Space Astronomy Centre (VilSpa) \\\hline
FL&First Look \\\hline
FLOP&FLoating-point OPeration \\\hline
FTE&Full-Time Equivalent \\\hline
GAIA&Global Astrometric Interferometer for Astrophysics (obsolete; now spelled as Gaia) \\\hline
GWP&Gaia Work Package \\\hline
HW&Hardware (also denoted H/W) \\\hline
ICD&Interface Control Document \\\hline
ID&Identifier (Identification) \\\hline
IDT&Initial Data Treatment (Image Dissector Tube in Hipparcos scope) \\\hline
ISO&International Organisation for Standardisation (Geneva, Switzerland) \\\hline
IT&Information Technology \\\hline
JD&Julian Date \\\hline
JDK&Java Development Kit \\\hline
LaTeX&(Leslie) Lamport TeX (document markup language and document preparation system) \\\hline
MAN&MANual \\\hline
MDB&Main DataBase \\\hline
OF&Object Feature (source packet) \\\hline
OSG&Operations Steering Group \\\hline
PA&Product Assurance \\\hline
PAP&Product Assurance Plan \\\hline
PDR&Preliminary Design Review \\\hline
PO&Partial Observation (of object in AF) \\\hline
PPN&Parametrised Post-Newtonian (formalism in General Relativity) \\\hline
PR&Progress Report \\\hline
QA&Quality Assurance \\\hline
RAM&Random Access Memory \\\hline
SADT&Structured (System) Analysis and Design Technique \\\hline
SCI&Schedule-Critical Item \\\hline
SCMP&Software Configuration Management Plan \\\hline
SDD&Software Design Document \\\hline
SDP&Supplementary Data Pattern \\\hline
SGS&Science Ground Segment \\\hline
SOC&System On a Chip \\\hline
SP&SPecification \\\hline
SPR&Software Problem Report \\\hline
SRR&System Requirements Review \\\hline
SRS&Software Requirements Specification \\\hline
SSS&System Software Specification \\\hline
STP&Software Test Plan \\\hline
STR&Software Test Report \\\hline
STS&Software Testing Specification \\\hline
SUM&Software User Manual \\\hline
SVN&SubVersioN \\\hline
SVTP&Software Verification Test Plan \\\hline
SW&Software \\\hline
TN&Technical Note \\\hline
TRB&Test Review Board \\\hline
TRR&Test Readiness Review \\\hline
UML&Unified Modeling Language \\\hline
URL&Uniform Resource Locator \\\hline
VV&Verification and Validation \\\hline
WBS&Work Breakdown Structure \\\hline
WP&Work Package \\\hline
\end{longtable} 
