\section{Data Management Senior Positions and Responsibilities}
LSST Data Management Managers and Staff
These individuals form the top level management of the DMO.
DM Project Manager
The DM Project Manager is responsible for the efficient coordination of all LSST activities and responsibilities assigned to the DMO. The DM Project Manager has the responsibility of establishing the organization, resources, and work assignments to provide DM solutions.  The DM Project Manager, serves as the DMO representative in the LSST Project Office and in that role is responsible for presenting DM initiative status and submitting new DM initiatives for approval consideration. Ultimately, the DM Project Manager, in conjunction with his / her peer Project Managers (Telescope, Camera), is responsible for delivering an integrated LSST system. The DM Project Manager reports to the LSST Project Manager. Specific responsibilities include:

\begin{itemize}
\item Manage the overall DM System
\item Define scope and funding for DM System 
\item Develop and implement the DM project management and control process, including earned value management
\item Approve the DM Work Breakdown Structure (WBS), budgets and resource estimates
\item Participate in identification of new DM partners 
\item Approve or execute as appropriate all DM outsourcing contracts 
\item Convene and/or participate in all DM reviews
\item Co-Chair the DM Leadership Team
\end{itemize}
DM Deputy Project Manager
The DM Deputy Project Manager, if this position is implemented, assists the DM Project Manager in the efficient coordination of all LSST activities and responsibilities assigned to the DMO.  Specific responsibilities are the same as the DM Project Manager, when delegated to the DM Deputy Project Manager by the DM Project Manager.
DM Project Scientist
The DM Project Scientist has ultimate responsibility for ensuring DMO initiatives provide solutions that meet the overall LSST scientific and technical requirements.  The DM Project Scientist must ensure correct specification of DM Scientific Requirements and proper translation of those requirements into derived information technology requirements and ultimately, into implemented solutions.  The DM Project Scientist must ensure that the DM subsystem is properly scoped and integrated within the overall LSST system.  The DM Project Scientist is also a member of the LSST Project Science Team (PST) and reports to the LSST Director. Specific responsibilities include:

\begin{itemize}
\item Responsible for the science deliverables of the DM System
\item Set requirements for the DMS that:
\end{itemize}
o Ensure that the design and operational flow of the data products meet the needs of the science community
o Ensure that the quality requirements of the data products will be / are being met by the DMS, with a particular emphasis on choice of appropriate application algorithms
\begin{itemize}
\item Set requirements for and assess/validate results of Data Challenges and other precursor experiments
\item Set requirements and assess/validate results for Data Releases
\item Convene and/or participate in all DM reviews
\item Co-Chair the DM Leadership Team and Science/Architecture Team
\end{itemize}
DM Science Quality and Reliability Engineering (SQuaRE) Leads
The DM SQuaRE Leads are the SQuaRE Lead Scientist and the SQuaRE Technical Manager.  The primary organizational responsibility for this Tucson-led group is to provide scientific and technical feedback to the LSST DM Manager that demonstrates LSST/AURA DM is fulfilling its responsibilities as charged by the NSF with regards to science quality and software/IT performance and reliability.
They are responsible for monitoring the reliability and maintainability of software developed by DM and the quality of the data products produced by the DM software in production. SQuaRE's activities span processes and environments for software development, integration test and distribution.  SQuaRE also assumes responsibility for delivering any work in this area, though in many cases this may involve effort across the DM team. 
As such, areas of activity include:
\begin{itemize}
\item Development of algorithms to detect and analyze quality issues with data
\item Infrastructure development to support the generation, collection, and analysis of data quality and performance metrics
\item DM developer support services to ensure DM is using appropriate tools to aid software quality
\item Support of publicly released software products, including porting and distributing it according to the scientific community?s needs.
\end{itemize}

In the event that SQuaRE identifies issues with the performance or future maintainability of the DM codebase, it brings them to the attention of the DM System Architect, who is ultimately responsible to decide who will address them and how. In the event that SQuaRE identifies issues with the quality of the data, it brings them to the attention of the DM Project Scientist. 
DM System Architect
The DM System Architect is responsible for ensuring that all elements of the DM systems, including infrastructure, middleware, applications, and interfaces, adhere to a documented, integrated architecture, consistent with the overall LSST systems architecture.  The DM System Architect reports to the DM Project Manager and DM Project Scientist. Specific responsibilities include:
\begin{itemize}
\item Responsible for the overall design integrity of the DMS as a system, including
\end{itemize}
o Internal and external interfaces and overall architecture
o Use of frameworks and off-the-shelf components
o Sound engineering practice
\begin{itemize}
\item Ensure that the DMS design is traceable to an meets the requirements set by the DM Project Scientist, through requirements/design models and technical reviews
\item Set standards for development, implementation, and documentation of the DMS, and ensure that they are being met by all DM participants
\item Participate in stakeholder and end user coordination and approval processes and reviews
\item Member of the LSST System Engineering Team
\item Co-chair the DM Science/Architecture Team
\end{itemize}


\section{Lead Institution Senior Positions}
Each Lead Institution has a Project Manager and Scientific/Engineering Lead, who jointly have overall end product responsibility for a broad area of DM work, typically a Work Breakdown Structure (WBS) Level 2 element. They are supervisors of the team at that institution.  Their roles and responsibilities are similar to the DM Project Manager, DM Project Scientist, and DM System Architect, and DM QA and Test Lead, but within the scope of work assigned to that institution.  These leaders are bound to acknowledge and implement direction from the DM leadership in all matters pertaining to the DM project.  The DM Project Manager and DM Project Scientist have direct input into the performance appraisals of the Institution Project Manager and Scientific/Engineering Lead. 

