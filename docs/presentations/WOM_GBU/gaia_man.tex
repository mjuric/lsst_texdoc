
\section{Management of Gaia}
\frame {\frametitle{  Communication } 
\begin{itemize}
  \item Internal communication: 
    \begin{itemize}
      \item Some say could have been better.
      \item Few focused working groups and working meetings.
      \item As for any project cost of entry for new people is very high {\color{blue} --- no obvious solution}.
    \end{itemize}
  \item ESA policy initially to reduce contact between DPAC and Astrium (who construct Gaia)
    {\color{red} not good}.
  \item External communication: 
    \begin{itemize}
      \item Perhaps could have had a better DPAC website.
      \item ESA PR also not great (ok as they point out they have a tiny fraction of NASA budget).
      \item {\color{blue}Publication policy was dealt with very late.}
    \end{itemize}
\end{itemize}
}


\frame {\frametitle{  Requirements Management } 
\begin{itemize}
  \item Some requirements at an appropriate level are very useful. 
    \begin{itemize}
        {\color{red}
      \item Many could have been better formulated.
    \item Separation of performance, software, and science requirements should have been clearer.}
  \end{itemize}
\item All requirements and test reports then ingested in the Information Management Tool \citep{2012SPIE.8449E..0GC}:
    \begin{itemize}
\item Automated testing provided much requirements verification.
\item Operations Rehearsals used to validate many requirements.
\item {\color{red} Could probably have put more effort in this earlier.}
  \end{itemize}
\item Cumbersome ESA Reviews are an unavoidable part of all this:
    \begin{itemize}
	\item  {\color{red}Just do it.}  - that means many expected documents ..
	\item  {\color{blue} Had to convince all collaborators to also support reviews.} 
	\item {\color{green} Projected cohesive unified team image }
	\item  on LSST  believe you have no choice in this - less formal reviews which is probably worse for you perhaps better for LSST goals
  \end{itemize}
\end{itemize}
}


\frame[allowframebreaks] {\frametitle{  Management and  science } 
All large projects, and especially science projects, have management {\em issues}.
\begin{itemize}
  \item In 2006 we had a big management training week for the DPAC management --- though sceptical
    to start most found it good:
	\begin{itemize}
	\item Despite this management support was still lacking.
        \end{itemize}
  \item Science project management is a little different, still books like
    \citep{handy1993understanding} are quite useful.
  \item Cyclic (Agile type) development seems well suited to science:
    \begin{itemize}
      \item We chose six month cycles --- probably too long (good for reporting).
      \item {\color{green} Some prototypes started very early} \citep{1999BaltA...8...57O}.
      \item  We have great simulations --- {\color{green}they started in 1998 }before Gaia was accepted:
        \begin{itemize}
          \item Still they always seemed to be a little behind what people wanted --- {\color{blue} we
          have no solution for that, could not start earlier.}
          \item Simulator fell out of the ECSS rigour --- testing etc\ldots
        \end{itemize}
      \item We did a lot of testing; {\color{blue} some tests were probably not appropriate in hindsight}.
      \item {\color{red} Despite aiming for test driven development --- NOT ENOUGH effort in
      testing and many systems only recently got continuous integration.}
    \end{itemize}
  \item Scientific institutes are not good at managing things like software projects (hard anyway):
    \begin{itemize}
      \item {\color{red} Interface control between software was insufficient } --- data model was not enough. {\color{red} perceive this in LSST now and Datamodel is not as rigorous}
      \item Perhaps ESA should have taken control of all critical software.
      \item ESA is stepping back from this type of role in future missions and was not
        totally happy about the level of involvement of ESAC in DPAC.
    \end{itemize}
  \item Industrial contracts for scientific software are very difficult:
    \begin{itemize}
      \item Did an experiment with this on Gaia very early on.
      \item XMM have their own woeful tale to tell.
	\item Science consortia need managers and engineers already in the early phases
    \begin{itemize}
	
        \item  Noted lack of {\em engineering} in many areas ---
          {\color{red} lack of software engineers in initial phases, many hires were astronomers}.
    \end{itemize}

    \end{itemize}



\item DPAC too collaborative? 
  \begin{itemize}
    \item DPAC was broadly inclusive --- CU leaders on paper had the chance to include or not groups
      and WPs but in fact no one was left out.
    \item This has lead to some inefficiencies --- perhaps we could be smaller and more focused.
    \item We possibly should have jettisoned some work packages, groups, and individuals early on.
    \item {\color{blue} In a proper phased approach some CUs could probably have started 2 years
    later with minimum presence at kick off} 
  \end{itemize}
\item Too flexible and too inflexible:
  \begin{itemize}
    \item There are many configuration control boards and other groups to manage change.
    \item {\color{blue}No one wants any change to anything --- {\em until the moment they want a change  then it
      should be immediate!}}
    \item {\color{red} We have found no solution to this.} 
  \end{itemize}
\end{itemize}
}





